\chapter*{Abstract}
\label{sec:abstract}
\addcontentsline{toc}{chapter}{Abstract}
Zur Behandlung von Arrhythmien stellt die kardiale Kryoablation eine Alternative zur gängigen Hochfrequenzablation dar. In Zusammenarbeit des Instituts für Elektrotechnik und Biomedizinische Technik (IEBE) der UMIT mit AFreeze enstanden in Vergangenheit verschiedene Computermodelle zur Simulation kardialen Kryoablation mittels Finite Elemente Methode (FEM). Aus diesen können Schlussfolgerungen auf den Ablationsprozess gezogen werden um die Prozesse der Eisballentwicklung und Friergeschwindigkeit besser zu verstehen. In bestehenden Modellen wird bisher von idealisierten Bedingungen ausgegangen. Diese Anwendungsvariationen sollen im Rahmen dieser Bachelorarbeit analysiert und anhand des CoolLoop\SymbReg{} Kathetermodells von AFreeze und des bestehenden Simulationsframeworks untersucht werden. 

So wird in dieser Arbeit die transmurale Tiefe des Eises in Abhängigkeit der Positionierung des Katheters analysiert. Dies geschieht aus verschiedenen Betrachtungsweisen heraus: Zum einen mit einem äquidistanten Katheterabstand und zum anderen mit einer Schieflage des Katheters. Auch die Auswirkungen unterschiedlicher Auflagewinkel werden in Bezug auf transmuraler Eisbildung und Vereisungsgröße untersucht. Gleiches erfolgt mit unterschiedlichem Verdrängungsmöglichkeiten des myokardialen Gewebes bei einer Kathetereindringung. Hier werden drei verschiedene Varianten verglichen. Um das Tauverhalten im Blut in Zukunft realitätsnäher gestalten zu können, werden verschiedene Anpassungen des Blutperfusionsterms in dieser Arbeit gegenübergestellt.

Abschließend zeigt sich, dass von den gewählten Parametern alle, bis auf die Gewebeverdrängung, deutliche Auswirkungen auf die Parameter der Vereisung aufweisen:
So nimmt mit steigendem Katheterabstand die Tiefe des Eises und das Eisvolumen im Myokard ab. Variiert man die Katheterschieflage, so zeigen sich auch hier Änderungen der Temperaturverteilung am Epikard in Abhängigkeit von den Abständen beider Katheterseiten zum Myokard. Die Zunahme des Auflagewinkel des CoolLoop\SymbReg{} Katheters führt zu einer unvorteilhafteren Temperaturverteilung am Epikard. In dieser Arbeit gelang die Änderung des Blutperfusionsterms, so dass dieser entsprechenden Beobachtungen in der Praxis besser entspricht.