\chapter{Closing remarks}
\label{sec:ClosingRemarks}
\section{Summary}
\label{sec:09_Summary}
Das erklärte Ziel dieser Arbeit stellte die Untersuchung von Idealisierungseffekten bei der in-silico Analyse der kardialen Kryoablation dar, unter Verwendung des CoolLoop Modells. Hierfür wurden, ausgehend von einem Grundmodell der bestehenden Literatur, fünf Parameter herausgegriffen und gezielt analysiert. Dies geschah mit Hilfe von eigens dafür entwickelten Template-Geometrie-Dateien, unter Verwendung des in der UMIT entstandenen Modellierungsframeworks. Die Simulation erfolgte mit einer von Michael Handler entwickelten Software und die Auswertung mittels ParaView.

\section{Outlook}
\label{sec:09_Outlook}
Experten gehen davon aus, dass in wenigen Jahren die punktuellen Ablationsverfahren auf Grund von Komplexit{\"a}t und l{\"a}ngeren Eingriffszeiten keine Anwendung mehr finden...