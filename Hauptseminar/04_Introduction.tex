\chapter{Introduction}
\label{sec:Introduction}
\section{Motivation}
\label{sec:04_Motivation}
Die häufigste Herzrhythmusstörung stellt in Deutschland das Vorhofflimmern dar, wobei davon auszugehen ist, dass sich die Anzahl der Erkrankungen in den nächsten 40 Jahren mehr als verdoppeln wird...

% ########################################################################################################
\newpage
\section{Classification in the technical background}
\label{sec:04_Backgroundclassification}
Seit über drei Jahrzehnten werden kryoablative Verfahren in der Rhythmuschirurgie angwandt. Doch obwohl bereits charakteristische Effekte der Kryoablation für den Zelltod in der Literatur beschrieben und auch entsprechende in-vivo und in-vitro Experimente durchgeführt wurden, existieren bis dato nur wenige Studien, welche diese Effekte mit Hilfe von Computermodellen orts- und zeitabhängig für kardiale Kryoablationsszenarien analysieren. Um den Prozess der Eisentwicklung und Friergeschwindigkeit besser zu verstehen, kommt es zum Einsatz von Simulationen der kardialen Kryoablation mittels Finite Elemente Methode (FEM), um daraus Schlussfolgerungen auf den Ablationsvorgang zu ziehen.


% ########################################################################################################
\newpage
\section{Aim of this study}
\label{sec:04_aim}
Diese Arbeit soll sich damit befassen, Anwendungsvariationen des CoolLoop\SymbReg{} Katheters zu untersuchen. Dies bezieht sich auf getroffene Annahmen in bisherigen Ablationsszenarien, welche lediglich ein vereinfachtes Bild des praktischen Einsatzes des CoolLoop\SymbReg{} Katheters darstellen, zum Beispiel ...